\documentclass[10pt]{article}

\usepackage{paper}

\setpapertitle{Finding Concentration Bounds for K-Means++}
\setauthor{Ritesh Baldva}{rb3447}
\addauthor{Gurpreet Singh}{gs3056}

\newcommand{\opt}[1]{#1_{\text{OPT}}}

\begin{document}
\makeheader%

\begin{abstract}
    Lloyd's algorithm~\citep{kmeans}, commonly known as the k-means algorithm, provides
    an intuitive sub-optimal algorithmic solution to the data clustering problem.
    The k-means algorithm has a number of variants and applications in different domains
    and is the most popular clustering algorithm in Machine Learning. Despite its
    simplicity and popularity, there is lack of information theoretic bounds on the
    clustering solutions. It is affected by initialization strategies, along with the
    contuinity and unboundedness of the loss function. In this survey, we peruse some
    recent strategies which try to find theoretic bounds for the k-means solution based 
    on assumptions on the underlying data distribution or modified initialization
    strategies.
\end{abstract}

\begin{psection}{Introduction}

    The data clustering problem that we deal with can be defined as minimizing the cost
    which is defined for a set of $n$ points in $\bR^d$ as the sum of squared l2 norm
    distance between each point and the closest center out of $k$ chosen centers. This,
    in general, is NP-hard.~\cite{kmeans} proposed a local search solution which begins
    with $k$ arbitrary centers and iteratively improves the clustering to greedily
    minimize the cost. The simplicity and the ability to parallelize the
    algorithm~\citep{parallel-kmeans} makes it one of the most popular clustering
    algorithms. However, due to the arbitrary initialization procedure and the poperties
    of the loss function, the algorithm suffers from unbounded cost estimates.

    In section 3, we explore some uniform deviation bounds derived for the k-means
    algorithm. The strategies discussed assume different conditions for the underlying
    probability distributions.

    The k-means++~\citep{kmeanspp} algorithm affords a well designed initialization
    strategy which helps us bound the expected cost within $\bigO{\log k}$ factor of the
    optimal cost. Despite the bounded results on expectation, the k-means++ strategy has
    some caveats that are highlighed in Section 4 along with some recent advances which
    try to tackle these issues. We also state the theoretical guarantees achieved by
    some of the recent seeding based strategies for k-means initialization without
    proof.

\end{psection}

\begin{psection}{Background and Notation}

    Define $d(\vx, C) = \min_{\vc \in C} \norm{\vc - \vx}_2$

\end{psection}

\begin{psection}{Uniform Deviation Bounds}
    % Moment-based Uniform Deviation Bounds fork-means and Friends
    If we use the Lloyd's algorithm for identifying the k-centres by approximating the
    k-means cost function, one problem that needs to be addressed is how well do the
    selected centers fit the underlying the data distribution. This problem refers to
    calculating to bounding the following quantity,
    \begin{equation}
        \label{moment:eq1}
        \left\vert \frac{1}{m} \sum_{j=1}^m \min_i \Vert x_j - p_i \Vert^2_2 -\int \min_i \Vert x - p_i \Vert^2_2 d\rho(x)   \right\vert  
    \end{equation}
    In the above equation~\ref{moment:eq1}, $\rho$ refers to the probability
    distribution of the data and $\set{p_i}_{i=1}^{k}$ refer to the set of selected
    centers. The paper establishes the above deviation to follow $\cO (m^{-1/2})$, where
    $m$ is the number of samples from the distribution data. \\ After running the
    algorithm, we obtain the cost on the samples, but evaluating the cost according to
    the probability measure can still make it unbounded. The idea is that the generally
    the samples will exist from areas of high density and the term $\int \norm{x -
    p}^2_2 d\rho(x)$ will be negligibly small for points which are far away from the
    centers, because of low density at those points. Thus, we can create balls around
    such centers and be able to move around within the balls to chose a new center and
    the effect of that new center from points far away will still be of the same order
    as previously. This can be seen easily with the help of the triangle inequality.
    Thus we have two tasks, bounding the deviations within the balls and the deviation
    for points which lie outside the balls.  Let's introduce the notation to follow the
    lemmas for finally bounding the deviation rates, which is the same as defined in the
    paper.

    \begin{enumerate}
        \item A convex differentiable function $f$, which for the hard k-means problem
            is, $f= \norm{x}^2_2$. Let's also assume that the function is
            $\alpha$-strongly convex and $\beta$-smooth. Let's also define the related
            Bregman divergence with respect to $f$, as $B_f(x,y)$. 
        \item Order-$p$ moment bound $M$ on $\rho$, i.e
            $\E[\rho]{ \norm{X - \E[\rho]{X}}^l} \leq M \,\, \forall l \in [1,p]$
        \item $P$ represents the set of centers and the corresponding cost of a single
            point $x$, with respect to $P$ and $f$, as $\phi_f(x; P) = \min_{p \in P}
            B_f(x,p)$. Then, $k$-means cost of a set of points with respect to a
            probability measure $\nu$ is $\E[\nu]{\phi_f(X; P)}$.  
    \end{enumerate}

    We first start with bounding the deviations between the $k$-th
    moments of the true measure and the empirical measure. This is done by first
    establishing a ball in the euclidean space which satisfies the following
    [Lemma A.6] in \citep{telgarsky2013moment}, where $\tau$ is a moment map,  
    \begin{equation}
        \label{moment:eq2}
        B = \set{x \in \cX : \norm{\tau(X)} \leq \para{M/\epsilon}^{1/(p - k)}}
    \end{equation}

    This results in the following bound on the moment on the true measure $\rho$,
    \begin{equation}
        \label{moment:eq3}
        \int_{B} \norm{\tau(x)}^k d\rho(x) \leq \epsilon
    \end{equation}

    The deviations between the $k$-th moments is derived through the Minkowski's
    inequality [Lemma A.7] in \citep{telgarsky2013moment} and a concentration bound
    established using bounded moments of order $p$ [Lemma
    A.3]\cite{telgarsky2013moment},
    \begin{equation}
        \label{moment:eq4}
        \abs{\int_B \norm{\tau(x)}^k d\hat{\rho}(x) - \int_B \norm{\tau(x)}^k d{\rho}(x)} \leq \sqrt{\frac{M' e p'}{2m}}\para{\frac{2}{\delta}}^{1/p'}
    \end{equation}

    which holds with probability at least $1 - \delta$ over draw of the sample size
    $m \geq p'/(M'e)$. In the above equation~\ref{moment:eq4}, $p'\geq 1$ and 
    $M' = 2^{p'}\epsilon$ and the relation holds for the ball with the maximum radius
    defined among all possible moment orders less than $p$.

    The main result for bounding the deviations in the k-means cost is defined int
    Theorem 3.2 of \citep{telgarsky2013moment}. The actual idea lies behind breaking the
    deviation into different terms using triangle inequality. From the complete space,
    we first reduce down to space (ball) where a decent amount of probability mass is
    present. Then we concentrate on the centers that are in and around this ball. From
    here we focus on a particular possible center of a disk that covers this ball. We
    try to bound the k-mean costs with respect to this ball locally through moment
    methods and use bracketing to bound the costs when reducing from the whole space to
    this actual ball. 
    \begin{align*}
        \abs{\int \phi_f(x; P) d\rho(x) - \int \phi_f(x; P)d{\rho}(x)}
        &\leq  \abs{\int \phi_f(x; P) d\rho(x) - \int_B \phi_f(x; P \cap C)d\hat{\rho}(x)} \\
        &\quad+ \abs{\int_B \phi_f(x; P \cap C) d\rho(x) - \int_B \phi_f(x; Q)d{\rho}(x)} \\
        &\quad+ \abs{\int_B \phi_f(x; Q) d\rho(x) - \int_B \phi_f(x; Q)d\hat{\rho}(x)} \\
        &\quad+ \abs{\int_B \phi_f(x; Q) d\hat{\rho}(x) - \int_B \phi_f(x; P \cap C)d\hat{\rho}(x)} \\
        &\quad+ \abs{\int_B \phi_f(x; P \cap C) d\hat{\rho}(x) - \int \phi_f(x; P )d\hat{\rho}(x)}
    \end{align*}

    In the above breakdown the ball $B$ represents the ball we concentrate on and $C$
    represents the region around this ball, which is considered viable to get centers
    from $P$. Finding the bracketing functions is easy, for the lower bound, it's
    trivially $0$ and for the upper bound we can split using the triangle inequality
    around the expected mass. 
    \begin{align*}
        l(x) &= 0 \\
        u(x) &= 4(\beta/2)\norm{x - \E[\rho]{X}}^2_2
    \end{align*}

    Notice that $u(x)$ can be used as moment function for the equation~\ref{moment:eq4}
    and with triangle inequality, thus used to bound the fifth term accordingly. We
    bound the first term with $\epsilon$ itself. For the third term, since $Q$ is a
    cover of the region surrounding the ball B, we use the following fact that,
    \begin{equation}
        \label{moment:eq5}
        \sup_{x \in B} \abs{\min_{p \in P} B_f(x, p)  - \min_{q\in Q}B_f(x,q)} \leq \epsilon
    \end{equation}
    which can be shown using properties of Bregman divergence only to have that, 
    \begin{equation}
        \label{moment:eq6}
        \abs{\int_B \phi_f(x; Q) d\rho(x) - \int_B \phi_f(x; Q)d\hat{\rho}(x)} \leq 4(\beta/2)R_C^2 \sqrt{\frac{1}{2m}\log\left(\frac{2 \abs{\cN}^k}{\delta}\right)}
    \end{equation}

    Plugging all the above bounds back, we get the bound for the deviation, where the
    value of $\epsilon$ is then chosen carefully to give the required bound for k-means
    problem. This results in a bound of the order $O(m^{-1/2 + \min(1/4, 2/p)})$, which
    results in $O(m^{-1/2})$ when $p \rightarrow \infty$. \\ In a more recent approach,
    \citet{approx-kmeanspp} show a consistent bound of $\bigO{m^{-1/2}}$. This is
    achieved by reformulation of the deviation problem from~\ref{moment:eq1}, to a
    functional form like below, 
    \begin{equation}
        \label{moment:eq7}
        \left\vert \frac{1}{m} \sum_{j=1}^m \min_i \Vert x_j - p_i \Vert^2_2 -\int \min_i \Vert x - p_i \Vert^2_2 d\rho(x)   \right\vert  \leq \frac{\epsilon}{2}\E[\rho]{\phi_f(X; P)} +  \frac{\epsilon}{2} \E[\rho]{\phi_f(X; \E[\rho]{X})}
    \end{equation}

    The authors claim in the above paper that along with bounding the deviation, the
    above formulation also allows for scale-invariance and unbounded solution space. The
    second term helps in controlling the scale-invariance through the variance term and
    the first term helps in generalising the bound uniformly for all sets of $k$-centers
    $Q$. The bound on the sample size, in the paper is obtained using in terms of the
    kurtosis bound on the distribution and a  generalization of VC dimension,
    psuedo-dimension. 
\end{psection}

\begin{psection}{Seeding for K-Means}

    The K-means algorithm is extremely sensitive to initialization. Almost all of the
    techniques which try to improve the convergence of the K-means strategy to a better
    optima revolve around finding better initialization means. This is known as seeding.
    In this section, we dicuss some of the earlier seeding strategies as well as the
    state-of-the-art popular strategies along with any information theoretic bounds they
    provide.

    In the native k-means algorithm, instead of finding a seeding, the data points are
    randomly assigned to a cluster and the seeding is chosen by computing the means of
    each data cluster. An improvement to this and one of the oldes methods is the
    Forgy Approach~\cite{forgy} where $k$ datapoints are chosen uniformly at random
    and used the initial seeds. A variant of this is to run the k-means algorithm $R$
    times each with a random initialization of the cluster means using the Forgy
    Approach. This is one of the most common approaches of seeding. This strategy helps
    in selecting a better seeding and somewhat reduces the sensitivity of the
    initialization, however, does not provide any definite results, nor help us derive a
    concentration bound on the convergence of the algorithm to a better optima.

    \citet{maxmin} proposed an almost deterministic strategy (Maxmin) to choose the
    seeds where only the first seed is chosed randomly. The MaxMin strategy is as
    follows; choose the first point uniformly at random from the dataset. At any
    timestep, choose a point from the dataset whose minimum distance is the maximum from
    any of the centroids (hence the name). The MaxMin strategy was shown to perform
    better than the original seeding strategy, the argument being that this strategy is
    based on inter- cluster distance optimization.~\cite{ka} proposed a completely
    deterministic variant of the MaxMin strategy which selects the seeds based on
    datapoints which improve the loss by the most amount.

    There are many researches which conduct an empirical comparison of the
    aforementioned approaches~cite{pena, celebi, helan, douglas}.~\cite{sami} provides
    a nice review on the different surveys and comparasons on some seeding strategies
    for k-means algorithms. None of the above methods, however, afford any theoretical
    guarantees on the learnt clustering when using Lloyd's Algorithm.

    Due to the arbitrary initialization of the cluster means, it is not possible to
    bound the loss even in expectation. In order to circumvent this,~\cite{kmeanspp}
    proposed an initialization strategy, called the k-means++ strategy, which allows
    us to bound the loss of the k-means algorithm in expectation. The k-means++
    initialization strategy is presented in algorithm \hyperlink{algo:1}{1}. This is
    one of the most common seeding strategies used for k-means. Most of the newly
    proposed strategies are based on k-means++. Before discussing these attempts, we
    dig a little deeper into the k-means++ strategy in the following text.
    
    \begin{psubsection}{K-Means++ Seeding}
    
        The idea of the k-means++ strategy is to sample the means as far away from each
        other as possible. This maximizes the coverage of the data points and, hence,
        bounds the cost of the clustering.
    
        \begin{algo}{K-Means++}
            \ditem{Input:} Data samples $Q = \set{\vx_1, \vx_2 \dots \vx_n} \sim P^n$ \\
            \ditem{Output:} Cluster means $C = \set{\vc_1, \vc_2 \dots \vc_k} \sim \bR^k$
            \\[1em]
            \ditem{Procedure:}
            \begin{enumerate}
                \item Uniformly at random pick $\vc_1$ from $\vx_1, \vx_2 \dots \vx_n$.
                \item Let $C = \set{\vc_1}$.
                \item For $i = 2 \dots k$, do
                    \begin{enumerate}
                        \item Sample a new center $\vc_i$ using $D^2$ weighting from the
                            dataset. $D^2$ weighting assigns a probability to each sample as
                            follows
                            \begin{align*}
                                \prob{C_i = \vx \pipe C} \eq \begin{cases}
                                    \frac{1}{Z} d^2(\vx, C) & \mt{if} \exists j\ \mt{s.t.} \vx_j = \vx \\
                                    0 & \mt{else}
                                \end{cases}
                            \end{align*}
                            where $Z$ is the normalization constant.
                        \item Set $C = C \cup \set{\vc_j}$
                    \end{enumerate}
                \item Return $C$ as the set of required $k$ centers.
            \end{enumerate}
        \end{algo}
    
        The bound on the expectation as presented in the original paper is given in
        Theorem~\ref{thm:kmeanspp-bound}
    
        \begin{theorem}[\cite{kmeanspp}]\label{thm:kmeanspp-bound}
            For any set of data points $Q = \set{\vx_1, \vx_2 \dots \vx_n}$ in $\bR^d$,
            we can find a clusering $C$ using the k-means++ strategy such that in
            expecation, the cost of the clustering is bounded within $\bigO{\log k}$
            factor of the optimal cost. More specifically, we have
            \begin{align*}
                \E[C]{\phi_C} \qle 8 \para{\log k + 2} \opt{phi}
            \end{align*}
        \end{theorem}
    
        Note that since k-means always only reduces the cost, if the initial clusters
        afford a bound on the expected cost, then the cost for the final clusters found
        with the k-means algorithm will also satisfy the same bound.
    
        Without going into the details of the proof, we present the main lemmas used to
        derive the bound in expecation and some interpretations behind them.
    
        \begin{lemma}
            Let $A$ be a cluster from optimal clustering $\opt{\cC}$ of the samples $Q$,
            and let $\cC$ be a clustering with one center chosen uniformly at random
            from $A$. Then, we have
            \begin{align*}
                \E{\phi_\cC(A)} \eq 2\ \opt{\phi}(A)
            \end{align*}
        \end{lemma}
    
        This lemma entails that for any set of points, a point chosen at random will
        always have bounded optimal cost in expctation. This establishes the base case
        of the k-means++ algorithm for single point clusters.
    
        \begin{lemma}
            Let $A$ be an arbirary cluster from the optimal clustering $\opt{\cC}$ and
            $\cC$ be some arbitrary clustering. If we add a random center to $\cC$ from
            $A$ sampled using $D^2$ weighting, then we have
            \begin{align*}
                \E{\phi(A)} \qle 8\ \opt{\phi}(A)
            \end{align*}
        \end{lemma}
    
        The above two lemmas combined tell us that if we pick a single point from a
        cluster either using $D^2$ weighting or uniformly at random, the expected loss
        is bounded by a factor of 8 from the optimal cost for that cluster. This
        essentially means that if we can pick one point out of each cluster, our
        expected cost will be only a constant times the optimal cost. However, this is
        obviously not always possible because we do not know the actual clustering and
        therefore ensuring each cluster has a representative in the initial means is not
        possible. The third lemma allows us to bridge this gap and provides a way for us
        to prove Theorem~\ref{thm:kmeanspp-bound}.
    
        \begin{lemma}
            Let $\cC$ be any arbitrary clustering we have chosen. Choose $u > 0$
            clusters from $\opt{\cC}$ that do not have any representative in the
            clustering $\cC$. Let $Q_u$ represent the set of points in these $u$
            clusters. Also let $Q_c = Q \setminus Q_u$. Now suppose we add $t \le u$
            random centers to $\cC$ sampled from $Q$ using $D^2$-weighting. Let $\cC'$
            denote the resulting clustering and let $\phi'$ denote the corresponding
            potential. Then $\E{\phi'}$ is bounded as follows
            \begin{align*}
                \E{\phi'} \leq (\phi(\cX_c) + 8\,\opt{\phi}(\cX_u)) \cdot (1 + H_t) + \frac{u - t}{u} \phi(\cX_u)
            \end{align*}
            where $H_t$ denotes the harmonic sum, \ie\ $H_t = \sum_{i=1}^t \frac{1}{i}$.
        \end{lemma}
    
        The intuition behind the above lemma is as follows. As discussed earlier, we
        want to choose one point for each cluster. The probability of choosing an
        uncovered cluster $A$ in the above setting would be $\phi_\cC{A} / \phi_\cC$. If
        the cluster $A$ is not covered, we would expect it's cost to be high, and
        therefore the probability of choosing a point from $A$ would be high as well.
        Moreover, if there is a point chosen from $A$, then the cost would be low and
        therefore there is lower probability of another point being chosen from $A$ with
        $D^2$ weighting.
    
        This means that by using $D^2$ weighting, we are promoting cluster diversity
        and, therefore, trying to cover as many clusters we can. This is also the idea
        behind the proof the above lemma, and subsequently contributes to the intuition
        and the proof of the K-means++ strategy as a whole.

        \citet{stream-kmeanspp} further extend the k-means++ strategy to build sampling
        strategies which reduce the original dataset to size $\bigO{k \log k}$ and
        affords $\bigO{1}$ approximation guarantees with a constant probability. 
        \citet{adaptive-kmeanspp} show that it is possible with an arbitrarily high
        probability a constant approximation guarantee based on the same $D^2$ sampling
        strategy. Both the approaches build on the $D^2$ sampling by using a divide and
        conquer approach to build bi-criteria approximation for the k-means problem.

        \citet{stream-kmeanspp} develop an alternate strategy called k-means\# which
        samples each centroid $3 \cdot \log k$ times and feeds these samples to the
        k-means++ algorithm to generate the final clusters. This divide and conquer
        strategy is a single pass streaming algorithm which affords the same order
        guarantees as k-means++.

        Even though k-means++ has been one of the most popular techniques for seeding,
        it is not devoid of problems. \citet{sami} suggests that k-means++ generally
        works well but is more suitable for small to mid-sized datasets. One of the
        reasons to not choose this strategy for large datasets is that each iteration
        of the seeding strategy requires $\bigO{nd}$ time. This means that we have to
        make $k$ passes of $\bigO{nd}$ time.

        One might wonder that each pass of the Lloyd's algortihm also takes $\bigO{nd}$
        time, however, it is possible to parallelize the K-means
        algorithm~\cite{parallel-kmeans}. K-Means++, on the other hand, cannot be
        parallelized as each iteration depends on the sample from the previous
        iteration. There have been several attempts to remedy
        this~\cite{scalable-kmeanspp, approx-kmeanspp, dist-kmeanspp, fast-kmeanspp}
        which we explore in the next subsection.
    
    \end{psubsection}

    \begin{psubsection}{Fast and Scalable Seeding with K-Means++}
    
        As pointed out in the previous subsection, k-means++ faces the issue of
        parallelization. \citet{scalable-kmeanspp} propose a sampling solution which
        resembles the approach suggested by \citet{stream-kmeanspp}. This sampling
        strategy proposed provides an additive approximation error along with an
        $\bigO{\log k}$ approximation factor in expectation.

        The idea behind the algorithm (given in algorithm box \hyperlink{algo:2}{2}) is
        to sample $k$ points in each iteration of the $D^2$ sampling strategy and do
        this for $\log n$ iterations. Over these generated samples, k-means++ is
        run and the cluster centers that are output are used as the initial seeds for
        the k-means algorithm.

        \begin{algo}[0.9\textwidth]{K-Means$\Vert$}
        
            \begin{description}
                \item[Input:] Data samples $Q = \set{\vx_1, \vx_2 \dots \vx_n} \sim P^n$
                \item[Output:] Cluster means
                    $C = \set{\vc_1, \vc_2 \dots \vc_k} \sim \bR^k$

                \item[Procedure:]
                    \begin{enumerate}
                        \item Uniformly at random, sample $\vc$ from $Q$
                        \item Initialize $C = \set{\vc}$ and corresponding clustering $\cC$
                        \item Let $\psi = \phi_\cC(Q)$
                        \item Do $\bigO{\log \psi}$ times
                            \begin{itemize}
                                \item Sample $\vc$ from $Q$ independently with
                                    probability
                                    \begin{align*}
                                        \prob{\vc = \vx \pipe C} \eq \begin{cases}
                                            \frac{1}{Z} d^2(\vx, C) & \mt{if} \exists j\ \mt{s.t.} \vx_j = \vc \\
                                            0 & \mt{else}
                                        \end{cases}
                                    \end{align*}
                                    for so weighting factor $l$
                                \item $C = C \cup C'$
                            \end{itemize}
                        \item For $\vc \in C$, set the weight of $\vc$ be the number of
                            points in $Q$ closer to $\vc$ than any other point in $C$
                        \item Recluster the weighted points in $C$ in to $k$ clusters
                            using k-means++ and return the obtained centers.
                    \end{enumerate}
            \end{description}
        
        \end{algo}

        \begin{theorem}[\citep{scalable-kmeanspp}]%
            \label{thm:scalable-bound}
            Let $\alpha = \texp{- \para{1 - \exp{-l/(2k)}}}$. If $C$ is the set of
            centers at the beginning of an iteration of algorithm hyperlink{algo:2}{2}
            and $C'$ is the set of centers added, then
            \begin{align*}
                \E{\phi_{C \cup C'}}
                &\qle 8 \opt{\phi} + \frac{1 + \alpha}{2} \phi_C
            \end{align*}
        \end{theorem}

        The above theorem is the main result shown by \citet{scalable-kmeanspp}. This
        algorithm allowed the runtime of the initialization strategy go down simply
        because of parallelization trading off with the computation cost of the
        process. The authors also note that the algorithm works well, empirically, even
        if the inner loop is run only a constant number of times.  Moreover, the
        convergence of the actual Lloyd's algorithm is shown to improve with
        k-means$\Vert$ seeding~\cite{scalable-kmeanspp}.
        
        \citet{dist-kmeanspp} argued that the bound in Theorem~cite{thm:scalable} does
        not scale with $l$ as the scaling factor can be at least $1/2$. In their work,
        \citet{dist-kmeanspp} improve the theoretical bound for k-means$\vert$ which
        depends on the variance of the underlying probability distribution. This bound
        is stated in Theorem~\ref{thm:dist}.

        \begin{theorem}[\citep{dist-kmeanspp}]%
            \label{thm:dist}
            Suppose if the points generated by the seeding strategy in algorithm
            \hyperlink{algo:2}{2} is represented as $C$. A clustering of $Q$ based on
            the centroids obtained from clustering $C$, say $\cC$ is at most a constant 
            factor from the optimal potential of the clustering of the original data 
            points $Q$ along with an additiver term that depends on the variance of the
            underlying probability distribution. More specifically, we have
            \begin{align*}
                \E{\phi_\cC(Q)}
                &\qle 2 \para{\frac{k}{el}}^t \var{Q} + 26 \opt{\phi}
            \end{align*}
        \end{theorem}

        According to Theorem~\ref{thm:dist}, using k-means++ on the subsampled points
        gives a $\log k$ factor approximation of the optimal potential with an additive
        error as showing in the equation.

        An alternate work~\cite{approx-kmeanspp} tried to replace $D^2$ sampling with an
        MCMC approximation (K-MC\textsuperscript{2}) which allows the runs to be
        parallel and requires fewer iterations than k-means++. The proposal distribution
        for the said MCMC is a uniform distribution over the sample points, therefore
        assigning $1/n$ to each data point. The acceptance probability for the chain is
        given as follows
        \begin{align*}
            \prob{\tfunc{accept}{x} \pipe x_{j - 1}}
                \eq \min \set{\frac{d^2(x, C)}{d^2(x_{j - 1}, C)}, 1}
        \end{align*}

        \citet{approx-kmeanspp} argue that the total variation distance between the
        MCMC sample distribution and the actual distribution decreases exponentially
        with the chain length, say $m$. With this argument, the authors prove a
        convergence bound as shown in Theorem~\ref{thm:approx}.

        \begin{theorem}[\citep{approx-kmeanspp}]
            Let $k > 0$ and $0 < \eps < 1$. Let $p++(C)$ be the probability of sampling
            a seeding $C$ using k-means++ and $p_{meme}(C)$ be the probability of
            seeding $C$ using K-MC\textsuperscript{2}. Then, we have
            \begin{align*}
                \norm{p_{meme} - p_{++}}_{TV} \qle \eps
            \end{align*}
            for a chain length $\bigO{\gamma' \log \frac{k}{\eps}}$ where
            \begin{align*}
                \gamma' \eq \max_{C \subset Q, \norm{C} \qle k} \max_{\vx \in Q} \frac{d^2(x, C)}{\sum_{\vx' \in Q} d^2(\vx', C')}
            \end{align*}
        \end{theorem}

        The above theorem proves that the samples generated using the
        K-MC\textsuperscript{2} are, in expectation, are close to the samples generated
        using k-means++. Therefore, we can achieve results close to k-means++ without 
        each iteration spending a complete pass over the whole dataset.

        \citet{fast-kmeanspp} extend these results to result some of the implicit
        assumptions made by K-MC\textsuperscript{2}. The authors argue that
        K-MC\textsuperscript{2} does not work well with heavy tailed distributions.
        Moreover, they refine the theoretical analysis to rid of theses assumptions and
        allow the algorithm to afford the same results up to a constant factor.

        There are countless other approaches which all offer strategies which afford
        certain kinds of convergence bounds. There hasn't been a successful research
        on provably converging to the optimal point up to a constant factor. We try
        to show a simple bound based on convergence inequalities for depenedent
        random variables for the k-means++ strategy in the next and concluding
        subsection.
    
    \end{psubsection}

    \begin{psubsection}{Deriving a Trivial Bound for K-Means++}
        
        Our objective is to derive a concentration bound on the initial cost of the
        following form.

        Suppose the cost of a clustering $C$ achieved by the seeding strategy given in
        k-means++ is represented as $\phi_C$. We want to show that with probability greater
        than $1 - \delta$, we have
        \begin{align*}
            \phi_C \qle \E{\phi_C} + \eps
        \end{align*}

        We have seen such concentration bounds often in information theory. However, most
        concentration inequalities assume the random variables over which we are computing
        the expectation to be independent. This is not satisfied in our case, since the
        sampling of the centers is done sequentially and the probability distribution
        depends on each of the centers sampled previously.

        In order to overcome this, we plan to use concentration bounds which do not assume
        independence between random variables. There have been a series of forays into
        deriving such concentration bounds \citep{marton1, marton2, samson, kontor, paulin}
        all along similar lines. A general result derived \citep{kontor} claims the
        following concentration bound.

        \begin{theorem}
            Let $X = \set{X_i}_{i = 1}^n$ be a set of dependent random variables all taking
            values in a countable space $\cS$ such that the set of random variables is a
            coordinate projection over the probability space $\para{\cS^n, \cF, \bP}$ where
            $\cF$ is the powerset of $\cS^n$ and $\bP$ is a probability measure on
            $\para{\cS^n, \cF}$. For an $l$-Lipschitz function $\varphi : \cS^n \ra \bR$
            (with respect to hamming distance), then for any $\eps > 0$, we have
            \begin{align*}
                \bP\{\abs{\varphi - \E{\varphi}} \ge \eps\} \qle 2\ \texp{-\frac{\eps^2}{2 n l^2 \norm{\Delta_n}_\infty^2}}
            \end{align*}
            where $\Delta$ is a matrix whose $i,j$ element is defined as follows
            \begin{align*}
                {(\Delta_n)}_{i, j} \eq \begin{cases}
                    1 & \mt{if} i = j \\
                    \eta_{i, j} & \mt{if} i < j \\
                    0 & \mt{else}
                \end{cases}
            \end{align*}
            $\eta_{i, j}$ is defined for all $1 \le i < j \le n$  as follows
            \begin{align*}
                \eta_{i, j} \eq \underset{\prob{X_{1:i} = \vy_{1:i-1}\bar{w}} > 0}{\underset{\prob{X_{1:i} = \vy_{1:i-1}w} > 0}{\max_{\vx_{1:i-1}, w, \bar{w}}}} \norm{\prob{X_{j:n} \pipe \vx_{1:i-1} w} - \prob{X_{j:n} \pipe \vx_{1:i-1} \bar{w}}}_{\text{TV}}
            \end{align*}
            where $\norm{\cdot}_{\text{TV}}$ signifies the total variation between two
            probability distributions.
        \end{theorem}

        Suppose instead of generating $k$ centers using the k-means++ strategy, we
        generate $m$ centers sequentially considering only the previous $k-1$
        centers for D\sups{2}-weighting. Let these set of centers be denoted as
        $\set{\mu_i}_{i=1}^m$. Then, define the average cost $\overline{\phi}$ as
        follows
        \begin{align}
            \label{eq:phi-bar}
            \overline{\phi} \eq \frac{1}{m-k+1} \sum_{i=1}^{m-k+1} \phi_{\mu_{i:i+k-1}}
        \end{align}

        Note that if we choose the minimum over each quantity in the RHS of
        equation~\ref{eq:phi-bar}, then that would be a lower bound on
        $\overline{\phi}$. Therefore, bounding $\overline{\phi}$ helps us bound the
        cost for one set of centers.

        The interesting thing about this is that in expectaion, the previous bound
        for k-means++ still holds. For the sake of brevity, we exclude the proof of
        this from this report and simply state it as a lemma.
        \begin{lemma}
            If we sample $m$ centers using the k-means++ strategy, then we have
            \begin{align*}
                \E{\overline{\phi}} \qle 8 \para{\log k + 2} \opt{\phi}
            \end{align*}
            where $\overline{\phi}$ is as defined in equation~\ref{eq:phi-bar}.
        \end{lemma}

        Note that the given series is a Markov chain with a memory of $k-1$ \ie\
        each timestep depends on the previous $k-1$ timesteps. Example 2.15
        from~\cite{paulin} discusses a concentration bound for exactly such an
        example and derives a bound as follows.

        Supppose we have a function $f$ that satisfies the bounded differences
        property with constant $c$ and a set of random variables $\set{X_i}_{i=1}^m$
        which follow a markov chain with memory $k$, then
        \begin{align*}
            \prob{f(\vX) - \E{f(\vX)} > \eps} \qle \texp{-\frac{\eps^2}{4kmc^2}}
        \end{align*}

        Now the function $\overline{\phi}$ satisfies the bounded differences
        property with $c = n \cdot D^2$ where $D$ is the diameter of the data points
        \ie\ the maximum euclidian distance between any two data points. Using this,
        we can write the following theorem.

        \begin{theorem}%
            \label{thm:approx}
            If we sample $m$ centers using the k-means++ strategy with each sample
            using the previous $k-1$ centers for D\sups{2}-weighting, we can say
            that with probability at least $1 - \delta$,
            \begin{align*}
                \overline{\phi} \qle \E{\overline{\phi}} + 2 \sqrt{(k - 1) \log{\frac{1}{\delta}}} \frac{n \cdot D^2}{\sqrt{m - k + 1}}
            \end{align*}
            where $\overline{\phi}$ is defined as in equation~\ref{eq:phi-bar}.
        \end{theorem}

        This bound, however, is trivial. The reason is that if $m = \Om{k}$, then
        the error term is of the order of $n \cdot D^2$, which is the upper bound of
        the cost and is, therefore, trivial. If, however, we choose $m = \Om{n^2}$,
        then we remove the dependency on $n$. This, however, means that the sampling
        will be of order $\bigO{kdn^3}$ which is large for even a moderately sized
        dataset. Even if somehow can sample this many centers, the bound could still
        be trivial if the diameter of each cluster is small but the clusters
        themselves are far apart. Therefore, such a bound does not help us bound the
        cost.

    \end{psubsection}

\end{psection}

\bibliographystyle{plainnat}
\bibliography{references}

\end{document}
